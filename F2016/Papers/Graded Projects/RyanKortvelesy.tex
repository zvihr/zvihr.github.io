\documentclass[12pt]{article}
\usepackage[margin=1in]{geometry}
\usepackage{pbox}
\usepackage{graphicx}
\usepackage{booktabs} % Top and bottom rules for table
\usepackage{amsfonts, amsmath, amsthm, amssymb}
\usepackage{longtable,array,color,xcolor}
\usepackage[colorlinks = true,
            urlcolor  = blue]{hyperref}

\newcommand\narrowstyle{\SetTracking{encoding=*}{-50}\lsstyle}

\pagestyle{empty}

%\usepackage[top=.6in,bottom=0.5in,left=0.5in,right=0.5in]{geometry}

\setlength{\parindent}{0pt}
%\usepackage{enumerate}
\usepackage[shortlabels]{enumitem}
\setlist{noitemsep}
%\setlist[enumerate]{topsep=0pt,itemsep=-1ex,partopsep=1ex,parsep=1ex}

\begin{document}
{\bf \large Math 320 -- Computer Methods in the Mathematical Sciences I.}\\

\vspace{3mm}

{\bf \large Project Title:} Neural Networks\\
\vspace{2mm}

{\bf \large Authors:} Ryan Kortvelesy\\

\vspace{3mm}

\section{Presentation}
{\bf \large Comments:}
\begin{itemize}
\item Nice brain explanation -- would have been improved with images.
\item Good exploration of Universal Approximation Theorem.
\item Bias image -- good!
\item ``So much chain rule'' :-)
\item Great that you walked through the code and implementation. When the
display was buggy because of the projector, you rolled nicely with the punches --good job!
\end{itemize}

{\bf \large Grade:} 9.5/10

Great presentation of a complicated but fascinating tool. 
Could perhaps have been a bit more polished, but overall very well done!


\section{Paper}

{\bf \large Comments:}
\begin{itemize}
\item The comparison to the brain should be taken with a grain of salt;
for example, the brain is certainly not a feedforward network.
\item \textit{Are these figures from somewhere? If so, citation is needed.} Found the citations at the end -- ideally mention this in the body of the text.
\item Well done including an example (p 4-5), but it would have been
better to be even more explicit. (e.g. actual real numbers for 
the inputs and weights)
\item Great description of bias!
\item I think Figure 8 is describing raising the number of neurons in
one layer (rather than the number of layers).
\item Good job going through different types of gradient descent.
\item Beautiful images on page 9!
\item Nice commenting on the code.
\end{itemize}

{\bf \large Grade:} 9.5/10

Great paper with some clearly sophisticated programming.
The organization \& presentation of the paper could have been neater, and
some of the technical detail was a bit off. On the whole, though, it was
an ambitious paper that did a tremendous job.


\end{document}
