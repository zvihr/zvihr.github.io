\documentclass[12pt]{article}
\usepackage[margin=1in]{geometry}
\usepackage{pbox}
\usepackage{graphicx}
\usepackage{booktabs} % Top and bottom rules for table
\usepackage{amsfonts, amsmath, amsthm, amssymb}
\usepackage{longtable,array,color,xcolor}
\usepackage[colorlinks = true,
            urlcolor  = blue]{hyperref}

\newcommand\narrowstyle{\SetTracking{encoding=*}{-50}\lsstyle}

\pagestyle{empty}

%\usepackage[top=.6in,bottom=0.5in,left=0.5in,right=0.5in]{geometry}

\setlength{\parindent}{0pt}
%\usepackage{enumerate}
\usepackage[shortlabels]{enumitem}
\setlist{noitemsep}
%\setlist[enumerate]{topsep=0pt,itemsep=-1ex,partopsep=1ex,parsep=1ex}

\begin{document}
{\bf \large Math 320 -- Computer Methods in the Mathematical Sciences I.}\\

\vspace{3mm}

{\bf \large Project Title:} Finite Difference Method\\
\vspace{2mm}

{\bf \large Authors:} Rongrong Liu\\

\vspace{3mm}

\section{Presentation}
{\bf \large Comments:}
\begin{itemize}
\item The images of displayed equations were blurry (Jpeg is not your friend!)
\item Nice idea to draw the grid on the board and to work through an example.
\item Great chalkboard work!
\item Not quite enough explanation of the heat equation background (could
have used some more detail of the problem, before the solution method.)
\end{itemize}

{\bf \large Grade:} 8/10

Very nice chalk talk, with great computation, though short on background
and context.


\section{Paper}

{\bf \large Comments:}
\begin{itemize}
\item Well done outlining the paper in the introduction.
\item Where are the images from? Citation should be made.
\item Nice job with the derivation of equations in Section 2.
\item Where does Laplace's equation come from? What kind of functions
$U$ satisfy this equation?
\item Couldn't the system of equations on p 5 - 6 have been solved
even without splitting the boundary into four parts? Does this way save
computation time?
\item Truncation error analysis is a good idea, but it's not clear
how to interpret your results. Some sentences about what these error
terms mean for the approximation would be very helpful.
\item The conclusion is nice, but the ideas you mention didn't appear in the
rest of the paper.
\end{itemize}

{\bf \large Grade:} 7.5/10

This paper has good computations and looks at a range of problems,
but there is not enough explanation or motivation.



\end{document}
