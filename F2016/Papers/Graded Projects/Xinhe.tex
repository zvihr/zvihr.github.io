\documentclass[12pt]{article}
\usepackage[margin=1in]{geometry}
\usepackage{pbox}
\usepackage{graphicx}
\usepackage{booktabs} % Top and bottom rules for table
\usepackage{amsfonts, amsmath, amsthm, amssymb}
\usepackage{longtable,array,color,xcolor}
\usepackage[colorlinks = true,
            urlcolor  = blue]{hyperref}

\newcommand\narrowstyle{\SetTracking{encoding=*}{-50}\lsstyle}

\pagestyle{empty}

%\usepackage[top=.6in,bottom=0.5in,left=0.5in,right=0.5in]{geometry}

\setlength{\parindent}{0pt}
%\usepackage{enumerate}
\usepackage[shortlabels]{enumitem}
\setlist{noitemsep}
%\setlist[enumerate]{topsep=0pt,itemsep=-1ex,partopsep=1ex,parsep=1ex}

\begin{document}
{\bf \large Math 320 -- Computer Methods in the Mathematical Sciences I.}\\

\vspace{3mm}

{\bf \large Project Title:} Fourier Transform and Voice Recognition.\\
\vspace{2mm}

{\bf \large Authors:} Xinhe Shan \\

\vspace{3mm}

\section{Presentation}
{\bf \large Comments:}
\begin{itemize}
\item Nice image for Fourier series.
\item Great demonstration in MATLAB! (Square wave)
\item Well done describing how Fourier series approach the Fourier transform.
\item Nice animation of voice comparison.
\item The definition of Fourier Transform and particularly the FFT were
a bit shortchanged.
\item Very nice demonstration with the volunteer!
\end{itemize}

{\bf \large Grade:} 9.5/10

Engaging and sophisticated presentation, with some neat demonstrations!


\section{Paper}

{\bf \large Comments:}
\begin{itemize}
\item With numbering of definitions etc, the paper becomes very readable.
\item Exposition may have benefited from an outline of the structure in the 
introduction.
\item What determines the choice of $\omega_0$?
\item Nice images for the Fourier series. 
\item Description of FFT makes it clear where the $\log_2N$ factor comes from
but not clear where $N$ comes from.
\item Robustness to noise is really well described.
\item The application from Voice Recognition is great -- I'm glad you figured out a good way to do this!
\end{itemize}

{\bf \large Grade:} 9.5/10

Fun paper with serious math, sophisticated computation, and relevant applications! Could have been a bit more polished, but all-in-all really great.


\end{document}
