\documentclass[12pt]{amsart}
\usepackage{fullpage}
\usepackage{pbox}
\usepackage{graphicx}
\usepackage{booktabs} % Top and bottom rules for table
\usepackage{amsfonts, amsmath, amsthm, amssymb}
\usepackage{longtable,array,color,xcolor}
\usepackage[colorlinks = true,
            urlcolor  = blue]{hyperref}
\usepackage{verbatim}
\usepackage{enumerate}
\newcommand\narrowstyle{\SetTracking{encoding=*}{-50}\lsstyle}

\setlength{\parindent}{0pt}


\DeclareMathOperator{\sign}{sign}

\newtheorem{bigthm}{Theorem}   % Numbered separately, as A, B, etc.
\newtheorem{thm}{Theorem}[section]   % Numbered within each section
\newtheorem{alg}[thm]{Algorithm}   % Numbered within each section

\newtheorem{cor}[thm]{Corollary}     % Numbered along with thm
\newtheorem{lem}[thm]{Lemma}         % Numbered along with them
\newtheorem{prop}[thm]{Proposition}  % Numbered along with them
\newtheorem{fact}[thm]{Fact}
\newtheorem{subfact}[thm]{Sub-Fact} 
\theoremstyle{definition} 
\newtheorem{defn}[thm]{Definition}   % Numbered along with thm
\newtheorem{conj}[thm]{Conjecture}        % Numbered along with thm
\newtheorem{remark}{Remark}   
\newtheorem{ex}[thm]{Example}        % Numbered along with thm
\newtheorem{exercise}[thm]{Exercise}        % Numbered along with thm
\newtheorem{rmk}[thm]{Remark}
\newtheorem{notation}[thm]{Notation}
\newtheorem{question}[thm]{Question}
\newtheorem{prob}[thm]{Problem}
\newtheorem{ass}[thm]{Assumption}
\renewcommand{\thenotation}{}  % to make the notation
                               % environment unnumbered
\newtheorem{terminology}{Terminology}
\renewcommand{\theterminology}{}  % to make the terminology
\newcommand{\Spec}{\mbox{Spec }}
\newcommand{\Proj}{\mbox{Proj }}
\newcommand{\Sym}{\mbox{Sym}}
\newcommand{\Hom}{\mbox{Hom}}
\newcommand{\Aut}{\mbox{Aut}}

\newcommand{\mc}{\ensuremath{\mathcal}}
\renewcommand{\skip}{\hspace{1cm}}

\newcommand{\EE}{\ensuremath{\mathbb{E}}}
\newcommand{\ZZ}{\ensuremath{\mathbb{Z}}}
\newcommand{\RR}{\ensuremath{\mathbb{R}}}
\newcommand{\PP}{\ensuremath{\mathbb{P}}}
\newcommand{\QQ}{\ensuremath{\mathbb{Q}}}
\newcommand{\FF}{\ensuremath{\mathbb{F}}}
\newcommand{\CC}{\ensuremath{\mathbb{C}}}
\newcommand{\NN}{\ensuremath{\mathbb{N}}}

\newcommand{\A}{\ensuremath{\mathbb{A}}}
\newcommand{\p}{\ensuremath{\mathfrak{p}}}
\newcommand{\m}{\ensuremath{\mathfrak{m}}}%\usepackage[breaklinks]{hyperref}
\usepackage{amssymb}
%\usepackage{fixltx2e}
%\usepackage[version=3]{mhchem}
\usepackage{fullpage}
\usepackage{parskip}
\usepackage{color}
\usepackage{soul}

\begin{document}

\title{Math 320: Class notes}
\maketitle


{\bf Roots - Chapter 6}

This chapter deals with open methods.

\section{Newton's Method}

\section{Secant Method}

Analyzing the convergence of the secant method.

Suppose we have a function with a simple root at $r$ and 3-times differentiable in a nbhd.
\begin{align*}
 x_{n+1} &= &x_n - \dfrac{f(x_n)(x_n - x_{n-1})}{f(x_n) - f(x_{n-1})}\\
 x_{n+1} - r &= &(x_n - r) - \dfrac{f(x_n)((x_n -r) -  (x_{n-1}- r))}{f(x_n) - f(x_{n-1})}\\
 E_{n+1} &= &E_n - \dfrac{f(x_n)(E_n - E_{n-1})}{f(x_n) - f(x_{n-1})}\\
 E_{n+1} &= &E_n - \dfrac{(f'(r)E_n + \frac{f''(r)}{2} E_n^2 + O(E_n^3))\cdot (E_n - E_{n-1})}{f'(r) (E_n - E_{n-1}) + \frac{f''(r)}{2}(E_n^2 - E_{n-1}^2)} + O(E_n^3-E_{n-1}^3) \\
 E_{n+1} &= &E_n - \dfrac{f'(r)(E_n - E_{n-1})(E_n + \frac{f''(r)}{2f'(r)} E_n^2 + O(E_n^3))}{f'(r)(E_n - E_{n-1})(1  + \frac{f''(r)}{2f'(r)}(E_n + E_{n-1}) + O(E_n^2+ E_n E_{n-1} + E_{n-1}^2))} \\
 E_{n+1} &= &E_n - \dfrac{(E_n + \frac{f''(r)}{2f'(r)} E_n^2 + O(E_n^3))}{(1  + \frac{f''(r)}{2f'(r)}(E_n + E_{n-1}) + O(E_n^2+ E_n E_{n-1} + E_{n-1}^2))} \\
 E_{n+1} &= &\dfrac{E_n(1  + \frac{f''(r)}{2f'(r)}(E_n + E_{n-1}) + O(E_n^2+ E_n E_{n-1} + E_{n-1}^2)) - (E_n + \frac{f''(r)}{2f'(r)} E_n^2 + O(E_n^3))}{(1  + \frac{f''(r)}{2f'(r)}(E_n + E_{n-1}) + O(E_n^2+ E_n E_{n-1} + E_{n-1}^2))} \\
 E_{n+1} &= &\dfrac{E_n  + \frac{f''(r)}{2f'(r)}(E_n^2 + E_{n-1}E_n) + O(E_n^3+ E_n^2 E_{n-1} + E_nE_{n-1}^2)) 
- (E_n + \frac{f''(r)}{2f'(r)} E_n^2 + O(E_n^3))}{(1  + \frac{f''(r)}{2f'(r)}(E_n + E_{n-1}) + O(E_n^2+ E_n E_{n-1} + E_{n-1}^2))} \\
 E_{n+1} &= &\dfrac{\frac{f''(r)}{2f'(r)} E_{n-1}E_n + O(E_n^3+ E_n^2 E_{n-1} + E_nE_{n-1}^2)}{(1  + \frac{f''(r)}{2f'(r)}(E_n + E_{n-1}) + O(E_n^2+ E_n E_{n-1} + E_{n-1}^2))} \\
 E_{n+1} &= &\frac{f''(r)}{2f'(r)} E_{n-1}E_n \left[\dfrac{1 + O(E_n^2/E_{n-1}+ E_n+ E_{n-1})}{1  + \frac{f''(r)}{2f'(r)}(E_n + E_{n-1}) + O(E_n^2+ E_n E_{n-1} + E_{n-1}^2)}\right] \\
\implies E_{n+1} &=& C E_nE_{n-1} 
\end{align*}

What is the order of convergence? Take $q$ such that
\[
\lim_{n \to \infty} \dfrac{|E_{n+1}|}{|E_n|^q} = C \implies \lim_{n \to \infty} \dfrac{|C_2 \cdot E_nE_{n-1}|}{|E_n|^q} = C \]
\[
\implies \lim_{n \to \infty} \dfrac{|E_{n-1}|}{|E_n|^{q-1}} = C/C_2 \implies \lim_{n \to \infty} \dfrac{|E_{n}|^{q-1}}{|E_{n-1}|} =C_2/ C
\]
\[\implies \lim_{n \to \infty} \dfrac{|E_{n}|}{|E_{n-1}|^{1/(q-1)}} = (C_2/C)^{1/(q-1)}
\]
But the order of convergence can only take one value, so:
\[
1/(q-1) = q \implies q^2 - q - 1 = 0 \implies q = \frac{1}{2}(1 \pm \sqrt(5))
\]
And the true value is the golden mean.

\end{document}
