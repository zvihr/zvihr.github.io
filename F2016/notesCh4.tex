\documentclass[12pt]{amsart}
\usepackage{fullpage}
\usepackage{pbox}
\usepackage{graphicx}
\usepackage{booktabs} % Top and bottom rules for table
\usepackage{amsfonts, amsmath, amsthm, amssymb}
\usepackage{longtable,array,color,xcolor}
\usepackage[colorlinks = true,
            urlcolor  = blue]{hyperref}
\usepackage{verbatim}
\usepackage{enumerate}
\newcommand\narrowstyle{\SetTracking{encoding=*}{-50}\lsstyle}

\setlength{\parindent}{0pt}


\DeclareMathOperator{\sign}{sign}

\newtheorem{bigthm}{Theorem}   % Numbered separately, as A, B, etc.
\newtheorem{thm}{Theorem}[section]   % Numbered within each section
\newtheorem{alg}[thm]{Algorithm}   % Numbered within each section

\newtheorem{cor}[thm]{Corollary}     % Numbered along with thm
\newtheorem{lem}[thm]{Lemma}         % Numbered along with them
\newtheorem{prop}[thm]{Proposition}  % Numbered along with them
\newtheorem{fact}[thm]{Fact}
\newtheorem{subfact}[thm]{Sub-Fact} 
\theoremstyle{definition} 
\newtheorem{defn}[thm]{Definition}   % Numbered along with thm
\newtheorem{conj}[thm]{Conjecture}        % Numbered along with thm
\newtheorem{remark}{Remark}   
\newtheorem{ex}[thm]{Example}        % Numbered along with thm
\newtheorem{exercise}[thm]{Exercise}        % Numbered along with thm
\newtheorem{rmk}[thm]{Remark}
\newtheorem{notation}[thm]{Notation}
\newtheorem{question}[thm]{Question}
\newtheorem{prob}[thm]{Problem}
\newtheorem{ass}[thm]{Assumption}
\renewcommand{\thenotation}{}  % to make the notation
                               % environment unnumbered
\newtheorem{terminology}{Terminology}
\renewcommand{\theterminology}{}  % to make the terminology
\newcommand{\Spec}{\mbox{Spec }}
\newcommand{\Proj}{\mbox{Proj }}
\newcommand{\Sym}{\mbox{Sym}}
\newcommand{\Hom}{\mbox{Hom}}
\newcommand{\Aut}{\mbox{Aut}}

\newcommand{\mc}{\ensuremath{\mathcal}}
\renewcommand{\skip}{\hspace{1cm}}

\newcommand{\EE}{\ensuremath{\mathbb{E}}}
\newcommand{\ZZ}{\ensuremath{\mathbb{Z}}}
\newcommand{\RR}{\ensuremath{\mathbb{R}}}
\newcommand{\PP}{\ensuremath{\mathbb{P}}}
\newcommand{\QQ}{\ensuremath{\mathbb{Q}}}
\newcommand{\FF}{\ensuremath{\mathbb{F}}}
\newcommand{\CC}{\ensuremath{\mathbb{C}}}
\newcommand{\NN}{\ensuremath{\mathbb{N}}}

\newcommand{\A}{\ensuremath{\mathbb{A}}}
\newcommand{\p}{\ensuremath{\mathfrak{p}}}
\newcommand{\m}{\ensuremath{\mathfrak{m}}}%\usepackage[breaklinks]{hyperref}
\usepackage{amssymb}
%\usepackage{fixltx2e}
%\usepackage[version=3]{mhchem}
\usepackage{fullpage}
\usepackage{parskip}
\usepackage{color}
\usepackage{soul}

\begin{document}

\title{Math 320: Class notes}
\maketitle


\section{Error - Chapter 4}

An algorithm is a set of rules that gives instructions for
a sequence of operations. Often takes some input and performs
a calculation.

An iterative algorithm is an algorithm that iterates (or repeats),
providing a better answer at each step.

\begin{defn}
Suppose an algorithm yields an approximate answer $\hat{x}$ for some
unknown $x$. \begin{enumerate} 
\item Error is defined as $E_t = x - \hat{x}$.
\item Relative error is $\epsilon_t = \dfrac{x - \hat{x}}{x}$.
\end{enumerate}

\end{defn}

\begin{defn}
Suppose an iterative algorithm yields a sequence of approximations 
$(\hat{x}_{k})_{k = 1,2,\ldots,n}$. Then we define: \begin{enumerate}
\item Approximate error is $E_a = x_n - x_{n-1}$ 
\item Approximate relative error: $\epsilon_a = \dfrac{x_n - x_{n-1}}{x_n}$.
\end{enumerate}
We can take a stopping criterion $\epsilon_s$ such that 
the algorithm stops when $|\epsilon_a| < \epsilon_s$.
\end{defn}

Error can also arise from roundoff errors in computation.

\subsection{How are numbers stored in computers}

\begin{defn}
A base-$n$ number system is a system of representing
rational numbers as a sum of powers of $n$.

The base-$10$ number system, or decimal system, uses
powers of $10$.
\end{defn}

\begin{ex}
134.57 is equal to
\[ 1 * 10^2 + 3 * 10^1 + 4 * 10^0 + 5 * 10^{-1} + 7 * 10^{-2}\]
\end{ex}

\begin{defn}
The signed magnitude method sets aside the first bit to denote
the sign, and the remaining bits denote the number. $1$ means a 
negative number. Since $\pm 0$ is redundant, the number $-0$ is
taken to be the number $-2^{n-1}$.
\end{defn}

\begin{defn}
Double precision floating point representation looks like
\[\pm s \times b^e. \]
$s$ is called the significand (or mantissa), 
$b$ is the base of the number system,
and $e$ is the exponent. $s$ is written with one (nonzero) digit to the
left of the decimal point.
\end{defn}

When $b = 10$, this is called scientific notation.
Computers use $b = 2$ with $11$ bits set aside for a signed
exponent, and $52$ bits for a mantissa.

\begin{defn}
An overflow error occurs when numbers are larger than the  allotted
maximum. An underflow error occurs if it is smaller than the minimum.
\end{defn}

\begin{defn}
Roundoff error happens in computations when the true answer
is rounded off losing some small $\epsilon$.

Machine epsilon is defined as the smallest value $\epsilon$ such
that $1 + \epsilon \neq 1$, i.e. $\epsilon$ is the smallest value
within our precision.
\end{defn}

If you were to add $.1 + .01$ in a number system with only one significant
bit in the mantissa, the second term would be rounded off.
Furthermore if a number does not have a finite binary expansion,
even simple operations can yield small roundoff errors that accumulate.

\begin{ex}
Consider $\frac{1}{10}$ in binary. Suppose we round off to 5 digits.
Perform the addition.
\end{ex}


\begin{defn}
An infinite series is the limit of $n$-th partial sums of a sequence
as $n \to \infty$.
\end{defn}

For example, geometric series.

\begin{defn}
Truncation error is the error made by approximating an infinite sum
by a finite sum.
\end{defn}

\begin{thm}[Taylor's theorem]
Let $k \geq 1$ be an integer, and let $f : \RR \to \RR$ be $k$-times
differentiable at $a$. Then
\[
f(x) = f(a) + \dfrac{f'(a)}{1!}(x-a) + \dfrac{f''(a)}{2!}(x-a)^2 + \cdots
+ \dfrac{f^{(k)}(a)}{k!}(x-a)^k + h_k(x) ( x- a)^{k}
\]
with $h_k(x) \to 0$ as $x \to a$.
\end{thm}

\begin{proof}
Use L'Hopital's rule $k$ times.
\end{proof}

Explicit formula for remainder:

\begin{thm}
Suppose $f$ is $k+1$ times differentiable on the open
interval $(x,a)$ and continuous on the closed interval, then the remainder term 
\[ R_k = \dfrac{f^{(k+1)}(x^*)}{(k+1)!} (x - a)^{k+1}.\]
where $x^*$ is a number in the open interval $(x_0,x)$.
\end{thm}

\begin{defn}
Let $f$ and $g$ be two functions of a real variable.
Then, $f(x) = O(g(x))$ as $x \to a$ if there are positive
numbers $M,\delta$ such that
$|f(x)| \leq M|g(x)|$ for all $x$ such that $|x -a| \leq \delta$.
Read aloud as 'f is big-O of g'.
\end{defn}

$R_k(x)$ is big-O of $(x-a)^{k+1}$.

\begin{cor}
Suppose we want an error bound for an approximation
$f(x)$ using a Taylor polynomial of degree $k$. If we
have a uniform bound on $f^{(k+1)}$, this gives a bound
on $R_k$.
\end{cor}

\begin{ex}
\[
\log(1+x) = x - \dfrac{x^2}{2} + \dfrac{x^3}{3} - \cdots
\]
Let us approximate $\log(1.5)$ using the first three terms of the the
Taylor series. $.5 - .125 + .0417 = .4167$. The remainder term
is
\[R_3 = \dfrac{f^{(4)}(x^*)}{4!} (.5)^4\]
The fourth derivative is strictly decreasing:
\[ f^{(4)}(x) = (-1)^4 4! (1+x)^{-4}  \leq 4!. \Rightarrow R_3 \leq .0625.\]

Therefore, $\log(1.5) \in (.3542,.4792)$. True answer $\approx .4055$.
\end{ex}



\end{document}
