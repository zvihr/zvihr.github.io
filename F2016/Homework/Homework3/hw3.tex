\documentclass[12pt]{amsart}
\usepackage[fullpage]{geometry}
\usepackage{fullpage}
\usepackage{pbox}
\usepackage{graphicx}
\usepackage{booktabs} % Top and bottom rules for table
\usepackage{amsfonts, amsmath, amsthm, amssymb}
\usepackage{longtable,array,color,xcolor}
\usepackage[colorlinks = true,
            urlcolor  = blue]{hyperref}
\usepackage{verbatim}
\usepackage{enumerate}
\newcommand\narrowstyle{\SetTracking{encoding=*}{-50}\lsstyle}

\setlength{\parindent}{0pt}

\begin{document}

\title{Math 320: Homework 3}
Due: September 30, 2016
\maketitle

Please read through chapters 5 and 6 in the textbook.
Answer the following questions. Please submit all code
and output with brief descriptions of what you are doing.

\vspace{5mm}

\begin{enumerate}

\item Implement the False Position Method (regula falsi) in
MATLAB. 
\begin{enumerate}
\item The method should be implemented as

\centerline{{\tt falsi(l,r,func,epsilon,maxSteps)}}

taking a bracket $[l,r]$, a function {\tt func}, an approximate 
relative error bound $\epsilon$, and a positive integer {\tt maxSteps}. 
The algorithm should terminate when
the approximate relative error goes below $\epsilon$ or when the number
of iterations exceeds {\tt maxSteps}.
The output should be the midpoint value most recently computed.

\vspace{3mm}

\item Use your implementation to compute:

\centerline{{\tt falsi(0, 3, @(x) x\textasciicircum 4 - 17, 10\textasciicircum (-5), 200)}}


\item Modify your {\tt falsi} code to output the sequence of
true relative errors and approximate relative errors at each iteration.
Plot both of these sequences for the function as above.

Let the $y$-axis be on a log scale.
\end{enumerate}
\vspace{2cm}

\item Implement the Secant method in MATLAB.
\begin{enumerate}
\item The method should be implemented as

\centerline{{\tt secant(x0,x1,func,epsilon,maxSteps)}}

taking two initial guesses $x_0$ and $x_1$, 
a function {\tt func}, an approximate 
relative error bound $\epsilon$, and a positive integer {\tt maxSteps}. 
The algorithm should terminate when
the approximate relative error goes below $\epsilon$ or when the number
of iterations exceeds {\tt maxSteps}.
The output should be the final approximation.

\vspace{3mm}

\item Evaluate the function for

\centerline{ {\tt secant(4,5,@(x) x\textasciicircum 3 - exp(x),10\textasciicircum (-6),100)}}

\vspace{3mm}

\item Evaluate the function for 

\centerline{{\tt secant(0,1,@(x) x\textasciicircum 3 - exp(x),10\textasciicircum(-6),100)}}

\vspace{3mm}

\item Incorporate a graphical component to your code, so that
the function {\tt secant} plots the graph of {\tt func}, as well
as the secant line segment computed at each iteration. Repeat the
computations above and display the associated graphs.

\end{enumerate}

\vfill
\pagebreak

\item Prove that Newton's method converges quadratically, given that
the function being studied is smooth, and the algorithm is converging to a
simple root.
{\bf Hint:} Estimate the error using Taylor's theorem with remainder
term.

\vspace{1cm}

\item You suspect that a bracket $[l,r] \subset \mathbb{R}$ contains a
multiple root of a smooth function $f(x)$ which is everywhere nonnegative. 
For each of the following
methods describe if it will work and provide reasons.
\begin{itemize}
\item Bisection Method for $f(x)$.
\item Incremental Search Method for $f'(x)$.
\item Newton's Method.
\item Secant Method.
\end{itemize}

\end{enumerate}


\end{document}
