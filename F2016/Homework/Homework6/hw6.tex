\documentclass[12pt]{amsart}
\usepackage[fullpage]{geometry}
\usepackage{fullpage}
\usepackage{pbox}
\usepackage{graphicx}
\usepackage{booktabs} % Top and bottom rules for table
\usepackage{amsfonts, amsmath, amsthm, amssymb}
\usepackage{longtable,array,color,xcolor}
\usepackage[colorlinks = true,
            urlcolor  = blue]{hyperref}
\usepackage{verbatim}
\usepackage{enumerate}
\newcommand\narrowstyle{\SetTracking{encoding=*}{-50}\lsstyle}
\newcommand{\RR}{\mathbb{R}}

\setlength{\parindent}{0pt}

\begin{document}

\title{Math 320: Homework 6}
Due: November 9, 2016
\maketitle

Please read through chapters 11, 12.1, and 13 in the textbook.
Answer the following questions. Please submit all code
and output with brief descriptions of what you are doing.

\vspace{5mm}

\begin{enumerate}

\item Section 11.2 discusses various norms on matrices.
In class, we reviewed three properties of any matrix norm $\rho$:
\begin{enumerate}
\item $\rho(a M) = a \rho(M)$ for scalar $a$ and matrix $M$.
\item $\rho( M + N) \leq \rho(M) + \rho(N)$ for $M,N$ matrices.
\item $\rho(M) = 0$ only if $M$ is the zero matrix.
\end{enumerate}
Prove that these properties are satisfied for 1) the column-sum
norm, and 2) the spectral norm. \\ 
{\bf Hint:} for the spectral norm,
use the induced norm definition: 
$||M||_2 = \sup(||Mx||_2 /||x||_2)$, where the
norm on the right-hand side is the Euclidean norm on vectors.

\vspace{5mm}

\item Problem 11.9.

\vspace{5mm}

\item Problem 12.2. You may use the textbook implementation of
Gauss-Seidel, but add a subroutine that at each iteration, plots the
first two coordinates of the approximation.
Display the plots for part (a) and part (b).

\vspace{5mm}

\item Consider a matrix $F$ which acts on a vector in
$\RR^2$ by mapping $(x_1,x_2)$ to $(x_2, x_1 + x_2)$.

\begin{enumerate}
\item Write $F$ down explicitly.
\item Let $v = [0,1]$. In a $2 \times 10$ matrix, 
display $F^kv$ for $k = 0, 10$.
\item What are the eigenvalues and eigenvectors of $F$?
Use this to write an explicit formula for $F^k$.
\item Use everything you have done so far to write a formula
for the $k$-th Fibonacci number, where $F_0 = 0$ and $F_1 = 1$.
\end{enumerate}
\end{enumerate}

\end{document}
