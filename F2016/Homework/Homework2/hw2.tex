\documentclass[12pt]{amsart}
\usepackage[fullpage]{geometry}
\usepackage{fullpage}
\usepackage{pbox}
\usepackage{graphicx}
\usepackage{booktabs} % Top and bottom rules for table
\usepackage{amsfonts, amsmath, amsthm, amssymb}
\usepackage{longtable,array,color,xcolor}
\usepackage[colorlinks = true,
            urlcolor  = blue]{hyperref}
\usepackage{verbatim}
\usepackage{enumerate}
\newcommand\narrowstyle{\SetTracking{encoding=*}{-50}\lsstyle}

\setlength{\parindent}{0pt}

\begin{document}

\title{Math 320: Homework 2}
Due: September 16, 2016
\maketitle

Please read through sections 4.1 - 4.3 in the textbook.
Then answer the following questions. Please submit all code
and output with brief descriptions of what you are doing.

\vspace{1cm}

\begin{enumerate}

\item Prove that a finite base-$n$ representation is unique.
In particular, show that if a number has a finite base-$n$ representation,
any other finite base-$n$ representation must be the same one.

\vspace{1cm}

\item (Problem 4.4) For computers, the machine epsilon
$\epsilon$ can also be thought of as the smallest number that
when added to one gives a number greater than $1$. An algorithm
based on this idea can be developed as
\begin{itemize}
\item[Step 1:] Set $\epsilon = 1$.
\item[Step 2:] If $1 + \epsilon$ is less than or equal to $1$, then
go to Step 5. Otherwise go to Step 3.
\item[Step 3:] $\epsilon = \epsilon/2$.
\item[Step 4:] Return to Step 2.
\item[Step 5:] $\epsilon = 2 \times \epsilon$.
\end{itemize}
Write your own M-file based on this algorithm to determine
the machine epsilon. Validate the result by comparing it with
the value computed with the built-in function {\tt eps}.

\vspace{1cm}

\item (Problem 4.11) The Maclaurin series expansion
for $\cos x$ is:
\[ \cos x = 1 - \dfrac{x^2}{2} + \dfrac{x^4}{4!} - \dfrac{x^6}{6!}
+ \dfrac{x^8}{8!} - \cdots \]
Starting with the simplest version, $\cos x = 1$, add terms one
at a time to estimate $\cos(\pi/4)$. After each new term is added,
compute the true and approximate percent relative errors.
Determine the true value. Add terms until the absolute value of
the approximate error estimate falls below an error criterion
conforming to two significant figures.

\end{enumerate}


\end{document}
