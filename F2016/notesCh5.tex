\documentclass[12pt]{amsart}
\usepackage{fullpage}
\usepackage{pbox}
\usepackage{graphicx}
\usepackage{booktabs} % Top and bottom rules for table
\usepackage{amsfonts, amsmath, amsthm, amssymb}
\usepackage{longtable,array,color,xcolor}
\usepackage[colorlinks = true,
            urlcolor  = blue]{hyperref}
\usepackage{verbatim}
\usepackage{enumerate}
\newcommand\narrowstyle{\SetTracking{encoding=*}{-50}\lsstyle}

\setlength{\parindent}{0pt}


\DeclareMathOperator{\sign}{sign}

\newtheorem{bigthm}{Theorem}   % Numbered separately, as A, B, etc.
\newtheorem{thm}{Theorem}[section]   % Numbered within each section
\newtheorem{alg}[thm]{Algorithm}   % Numbered within each section

\newtheorem{cor}[thm]{Corollary}     % Numbered along with thm
\newtheorem{lem}[thm]{Lemma}         % Numbered along with them
\newtheorem{prop}[thm]{Proposition}  % Numbered along with them
\newtheorem{fact}[thm]{Fact}
\newtheorem{subfact}[thm]{Sub-Fact} 
\theoremstyle{definition} 
\newtheorem{defn}[thm]{Definition}   % Numbered along with thm
\newtheorem{conj}[thm]{Conjecture}        % Numbered along with thm
\newtheorem{remark}{Remark}   
\newtheorem{ex}[thm]{Example}        % Numbered along with thm
\newtheorem{exercise}[thm]{Exercise}        % Numbered along with thm
\newtheorem{rmk}[thm]{Remark}
\newtheorem{notation}[thm]{Notation}
\newtheorem{question}[thm]{Question}
\newtheorem{prob}[thm]{Problem}
\newtheorem{ass}[thm]{Assumption}
\renewcommand{\thenotation}{}  % to make the notation
                               % environment unnumbered
\newtheorem{terminology}{Terminology}
\renewcommand{\theterminology}{}  % to make the terminology
\newcommand{\Spec}{\mbox{Spec }}
\newcommand{\Proj}{\mbox{Proj }}
\newcommand{\Sym}{\mbox{Sym}}
\newcommand{\Hom}{\mbox{Hom}}
\newcommand{\Aut}{\mbox{Aut}}

\newcommand{\mc}{\ensuremath{\mathcal}}
\renewcommand{\skip}{\hspace{1cm}}

\newcommand{\EE}{\ensuremath{\mathbb{E}}}
\newcommand{\ZZ}{\ensuremath{\mathbb{Z}}}
\newcommand{\RR}{\ensuremath{\mathbb{R}}}
\newcommand{\PP}{\ensuremath{\mathbb{P}}}
\newcommand{\QQ}{\ensuremath{\mathbb{Q}}}
\newcommand{\FF}{\ensuremath{\mathbb{F}}}
\newcommand{\CC}{\ensuremath{\mathbb{C}}}
\newcommand{\NN}{\ensuremath{\mathbb{N}}}

\newcommand{\A}{\ensuremath{\mathbb{A}}}
\newcommand{\p}{\ensuremath{\mathfrak{p}}}
\newcommand{\m}{\ensuremath{\mathfrak{m}}}%\usepackage[breaklinks]{hyperref}
\usepackage{amssymb}
%\usepackage{fixltx2e}
%\usepackage[version=3]{mhchem}
\usepackage{fullpage}
\usepackage{parskip}
\usepackage{color}
\usepackage{soul}

\begin{document}

\title{Math 320: Class notes}
\maketitle


\subsection{Roots - Chapter 5}

Let's solve the equation $f(x) = 0$.

\section{Bracketing Methods}

\begin{defn}
Bracketing methods are any root-finding algorithms that
start with an interval containing a root.
\end{defn}

\begin{alg}[Incremental search] INPUT: function $f$, search interval $[l,r]$.
\begin{enumerate}
\item Divide the interval into $N$ sub-intervals, with points
$l = x_0 < \ldots < x_N = r$ evenly spaced in the interval.
\item Compute $f(x_i)$ for all $i$.
\item List all $i$ for which $\sign(f(x_i)) \neq \sign(f(x_{i+1})).$
\end{enumerate}
OUTPUT: List of brackets $\{(x_i,x_{i+1})\}$ flagged in step 3.

\end{alg}


\begin{question}[Accuracy]
Does this algorithm catch all the roots? 
What roots will this algorithm miss? Will we ever
catch extra roots?
\end{question}
It misses roots when $N$ is not large enough.\\
It does not catch roots whose neighborhood all has
one sign.\\
If a bracket is flagged, there will be a root in it. Why?

\begin{thm}[Intermediate Value Theorem] If $f$ is a continuous
function on $[a,b] \subset \RR$, 
$f(a) < 0$, and $f(b) > 0$; then there exists
a point $c \in [a,b]$ such that $f(c) = 0$.
\end{thm}

\begin{question}[Complexity]
How long does the algorithm take?
\end{question}
It evaluates the function $N+1$ times, and checks the sign on
each value. The complexity is $O(N)$.

\begin{question}[Error]
How close to the true roots will the roots
found by our algorithm be?
\end{question}
Take the approximation of a given root to be in the
middle of the flagged bracket. 
Our error is then bounded by $(l-r)/2N$.


---------

\subsection{Bisection method}

The bisection method is an iterative form of the
incremental search.

\begin{alg}[Bisection Method] INPUT: function $f$, 
search interval $[l,r]$ such that $\sign(f(l)) \neq \sign(f(r))$,
desired precision $\epsilon$.
\begin{enumerate}
\item If $l-r < 2\epsilon$, stop.
\item Divide the interval into $2$ equal sub-intervals: $l < m < r$.
\item Compute $f(m)$.
\item If $\sign(f(m)) = \sign(f(l))$, repeat with interval $[m,r]$;
else, repeat with interval $[l,m]$.
\end{enumerate}
OUTPUT: Final bracket endpoints.

\end{alg}

{\bf Accuracy:} Catches exactly one root.

{\bf Complexity:} Will stop when $l-r < 2\epsilon$ since $l-r$ shrinks
by half every time, the length of the interval at iteration $N$ is 
$(1/2)^N (l-r)$.
\[
(1/2)^N (l-r) < 2 \epsilon \iff -N +  \log_2(l-r)-\log_2(2\epsilon) 
\]

{\bf Error:} By design the error is at most $\epsilon$.


----

\subsection{False-Position Method}

The false-position method resembles the bisection method,
but differs in how to select the midpoint. 
For false-position, you take the value of 
\end{document}
