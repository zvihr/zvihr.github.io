\documentclass[12pt]{article}
\usepackage[margin=1in]{geometry}
\usepackage{pbox}
\usepackage{graphicx}
\usepackage{booktabs} % Top and bottom rules for table
\usepackage{amsfonts, amsmath, amsthm, amssymb}
\usepackage{longtable,array,color,xcolor}
\usepackage[colorlinks = true,
            urlcolor  = blue]{hyperref}

\newcommand\narrowstyle{\SetTracking{encoding=*}{-50}\lsstyle}

\pagestyle{empty}

\usepackage[top=.6in,bottom=0.5in,left=0.5in,right=0.5in]{geometry}

\setlength{\parindent}{0pt}

\begin{document}
{\bf \large Math 320 -- Computer Methods in the Mathematical Sciences I.}\\
\section{Final Paper Grading Rubric.}

\begin{enumerate}
\item {\bf Clarity.} Does the paper present your ideas readably, clearly,
and in an organized way?
\item {\bf Level of difficulty.} Does the paper represent a sincere
effort to explore in depth the topic at hand?
\item {\bf Accuracy.} Are the code and mathematical content presented
in the paper correct?
\item {\em For group projects:} {\bf Teamwork.} Were both members 
of the group involved in producing the research and document?

\end{enumerate}

General tips:

\begin{enumerate}
\item Labeled and numbered definitions, theorems, remarks, and algorithms 
can go a long way in making your writing clearer and more organized.
\item A block of code in your paper should ideally be set apart by
lines or indentation, and should be in a different font.
\item If you are including equations or code in the text of your paper,
make sure to explain it in the body of the text using complete sentences.
\end{enumerate}

\section{Final Presentation Grading Rubric.}

\begin{enumerate}
\item {\bf Clarity.} Does your presentation clearly and
in an organized way, present your ideas?
\item {\bf Engagement.} Does the presentation engage the audience
in your topic?
\item {\bf Content.} Is the material included in the presentation 
valid and relevant?
\item {\em For group projects:} {\bf Teamwork.} Were both members 
of the group involved in the presentation?
\end{enumerate}

General tips:

\begin{enumerate}
\item Multiple forms of communication are ideal: for example, live
demonstrations of \\ algorithms, images, videos, and audio can all be
useful in the right context.
\item Assume the audience knows as little as you did at the beginning
of your research.
\end{enumerate}
\end{document}
